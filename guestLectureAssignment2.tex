\documentclass[]{article}
\usepackage{amsmath}
\usepackage{amsfonts}
\usepackage{amssymb}
\usepackage{hyperref}
\usepackage[margin=0.75in]{geometry}
\usepackage{savetrees}

%opening
\title{Report on Guest Lecture "Designing Green Communication Systems for Smart and Connected Communities via Dynamic Spectrum Access"}
\author{Nima Seyedtalebi}

\begin{document}

\maketitle

\begin{abstract}
Smart and connected communities (SCC) promise to improve the quality of life for residents while decreasing their environmental burden. Smart devices are abundant  but connectivity has not grown as quickly because of the infrastructure costs involved. Network service providers are unlikely to build the necessary infrastructure for SCC because doing so is not profitable. Moreover, operational costs increase as the network grows so even if the infrastructure were in place, the costs of operating and maintaining such a network must be considered. To actualize the potential in SCCs, we must develop networks that are more energy-efficient, cheaper, and easier to administer. The solution presented in "Designing Green Communication Systems..." is to match application requirements with physical network characteristics to increase energy and spectral efficiency. This is achieved by using dynamic spectrum access (DSA) and a weighted, undirected graph that encodes both the underlying physical network structure and characteristics of interest like energy cost or latency. The graph is constructed so that finding the shortest path through it given some constraints corresponds to finding the most efficient way to send a message through the physical network. A new algorithm is proposed to find such a path.
\end{abstract}

\section{Description}
Real examples of smart and connected communities are uncommon, yet mobile and wireless IP-enabled devices are widespread. According to the Cisco Visual Networking Index, global Internet Protocol (IP) traffic will increase 127-fold\footnote{compared to 2005} and traffic from wireless and mobile devices will comprise over 63 percent of global IP traffic by 2021. In 2016, wireless and mobile devices accounted for 49\% of global IP traffic.\cite{cisco} If wireless and mobile devices are so dominant, why not simply use the existing infrastructure to realize the SCC vision? Building new infrastructure is prohibitively expensive, and the existing infrastructure and protocols are too inefficient to support sustainable SCCs. Networks for sustainable SCCs must be cost-effective, energy-efficient, and provide Quality of Service (QoS) controls.\cite{shah}

Current wireless network offerings can be grouped into short-range and long-range. Short-range options like WiFi and Bluetooth use an unlicensed frequency band of the electromagnetic spectrum. They do not work well for longer distances because many devices are required. Long-range options like 3/4G, GSM, and LTE are inefficient and use a part of the spectrum that require a license. The license holder may only use the parts of the spectrum that it is licensed for. The number of usable bands is limited and demand for wireless and mobile networks is increasing steadily, so there is an incentive to overprovision to allow for growth since licenses may not be available in the future. Underprovisioning is also possible e.g. when a provider's subscriber base grows too quickly to obtain additional frequency bands. Both are bad, and there are further problems beyond those such as interference. Neither the existing short nor long-range technologies are sufficient alone, but perhaps they can be combined to overcome the limitations?

A technique called dynamic spectrum access (DSA) makes it possible to bridge the gap between the short and long-range systems by allowing devices to use different parts of the spectrum. In the United States, some bands like LTE \footnote{For licensed bands, secondary users must not cause interference for the primary licensee. LTE is a notable example of a licensed band where this is allowed} allow for DSA to be used. If a DSA-enabled device can access both WiFi networks and LTE networks, it can be used as a bridge between them. Very minimal new infrastructure is required and we can reuse existing networks, so this address our cost concerns. 
The traffic in an SCC network comes from a variety of devices, so there will be many different kinds of traffic. These disparate systems and messages do not share the same requirements. For example, losing a few packets of streaming video may be acceptable whereas file transfers require more reliability. DSA can address both our efficiency and QoS concerns by choosing which band(s) to use based on what is being sent. For example, a traffic sensor might not send or receive as much data as a mobile phone that is streaming video, so it does not need as much bandwidth. Since it is not interactive, perhaps the sensor also has relaxed latency requirements. A DSA-enabled device could send the sensor data on a slow, cheap channel and the video on a faster one. With DSA, we can choose the best available channel based on the requirements for each kind of traffic, which allows us to address both efficiency and QoS concerns. It should be noted that DSA-enabled networks are currently not suitable for real-time applications.

There remains an important question to be answered: how do we find the best channel to use for a given message and requirements? We can model a DSA-enabled network as a weighted, undirected graph where each edge represents a communication channel\footnote{That is, a usable connection between two nodes that uses a specific part of the specturm} and each node represents a networked device. If we label the edges with the band the channel uses and weights representing latency and energy cost, then finding the best band amounts to finding the shortest path that meets the constraints for the traffic. This is an instance of the restricted shortest path problem\footnote{Restricted shortest path is known to be NP-Hard} The presentation introduces a new method that uses dynamic programming (DP) and combines the Floyd-Warshall algorithm for finding all-pair shortest-paths with a DP algorithm for solving the Knapsack problem. This makes intuitive sense - the energy cost corresponds to path weights, latency corresponds to the weight of each item to be placed in the eponymous hypothetical knapsack, and latency constraints correspond to the maximum weight said knapsack can carry.
\section{Discussion}
The energy cost and latency must be known in advance to run the algorithm. How will we know these things? Figuring them out won't be free.
What does the device support look like? What things can actually do DSA?
What about DSA for real-time applications?
\bibliographystyle{plain}
\bibliography{guestLectureAssignment2}

\end{document}





