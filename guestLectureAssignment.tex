\documentclass[]{article}
\usepackage{amsmath}
\usepackage{amsfonts}
\usepackage{amssymb}
\usepackage{hyperref}
\usepackage[margin=0.75in]{geometry}
\usepackage{savetrees}

%opening
\title{Report on Guest Lecture "Stochastic Geometry and Wireless Networks"}
\author{Nima Seyedtalebi}

\begin{document}

\maketitle

\begin{abstract}
Stochastic geometry is the study of random patterns in space. Using these tools, we can answer questions in a diverse array of fields ranging from astronomy to forestry. This report is structured like the presentation mentioned in the title. We begin with a fundamental and important kind of point process called the stationary Poisson Point Process (PPP). This fundamental kind of point process has properties that make it useful for building mathematical models.The points in a point process can be partitioned into two sets based on some given probability and possibly other conditions in a transformation called \textit{thinning}\footnote{The name makes much more sense if "thinning" is understood as a reduction of density.} Point processes have applications in simulating and planning wireless networks. Modern cellular networks are becoming more heterogeneous and unplanned, so point processes are useful to simulate such and study the effects of changes. Interference is one of the main problems facing cellular networks today, and by modeling such networks as point processes, one can find the probability distribution of the signal+interference to noise ratio or investigate the effect of adding a new cell tower. We can consider networks where all of the base stations are the same (homogeneous) or where they are different (heterogeneous).
\end{abstract}

\section{Description}
Stochastic geometry extends the study of randomness to patterns in space. For this report, we will consider a specific class of mathematical objects defined on a subset of $\mathbb{R}^2$. A \textit{point process} in $\mathbb{R}^2$ as a random variable taking values in the space $\mathbb{N}$, where $\mathbb{N}$ is the set of all sequences $\phi \subset \mathbb{R}^2$ such that $\mathbb{N}$ is both finite and simple\cite{kham}\footnote{A sequence is simple if $\forall x_i,_j x_i \neq x_j$ if $i \neq j$}. Random variables are associated with probability distributions. Point process can be classified by the distribution of their random variables. For example, we have the simple example of a single uniformly-distributed point, the binomial point process, and the Poisson point process. A point process can be completely described by its void probability distribution. The void probability is the probability that no points from the process appear in some subset of the plane $K \subset \mathbb{R}^2$\cite{ppp}.
Some point processes are \textit{stationary}, which means their distributions do not change under translation.
Stationary point processes are desirable from an analytical perspective because they are the "same" everywhere. One important kind is the stationary Poisson point process (sPPP).

The sPPP is important bIf we think of coloring the points one by one, the probability that the next point is red is always the same.ecause it is both fundamental and useful by itself. Stationary Poisson point processes can be characterized by density, or the expected number of points in $B \subset \mathbb{R}^2$ divided by $|B|$. A stationary Poisson point process has a density equal to $\lambda$, the mean of the associated Poisson distribution. Since the distribution of a stationary process does not change with respect to position in space, we can completely describe a PPP using only $\lambda$. Another important property of the PPP is that the points are all independent of each other\footnote{More formally, given sets $A$ and $B$, if $A \cap B = \emptyset$ then the number of points in $A$ and $B$ is independent.}. The points in any such subset are independent and uniformly distributed over that set when we condition on the number of points in the subset. This means that any such subset is a binomial point process (BPP) and that the Poisson point process is a generalization of the BPP. Recall too that a BPP is a collection of $n$ independent uniformly-distributed point processes with a single point each. We can construct more complex point processes by combining and generalizing the fundamental ones.

Stationary Poisson point processes can be transformed by an operation called \textit{thinning}. As an example, consider a PPP $\phi$ with density $\lambda$. If we color some points red with probability $p$ and the others blue with probability $1-p$, we are left with two new PPP $\phi_r$ and $\phi_b$, the union of which is the original process $\phi$. The density of these processes is $\lambda p$ and $\lambda (1-p)$.
If $0 < p < 1$, then $\phi_r$ and $\phi_b$ must have a density smaller than $\lambda$ and are both "thinner" (less dense) than the original process. This type of thinning is called "independent thinning" because each point is colored independently of the others. If we think of coloring the points one by one, the probability that the next point is red is always the same. In a dependent thinning process, the probability of coloring the next node red is affected by some property of the other nodes, so the probability at each node depends on the others.

Applications of stochastic geometry abound. Its tools can be applied to problems in diverse areas such as forestry, astronomy, and engineering. Stationary Poisson point processes are widely used to model systems where the points are randomly located. For example, we can model a modern cellular network using a PPP. The points in the process represent base stations to which mobile devices connect.  It may seem strange at first that we do not specify the exact location of each base station, but this reflects the reality of modern cellular networks: they are often unplanned and include different kinds of base stations with varying capabilities.

One of the main challenges facing cellular network designers and operators is interference. We can characterize the amount of interference a given user experiences using a dimensionless quantity called the signal to interference-and-noise ratio, or SINR. If we add a few further assumptions (e.g. "mobile users always connect to the nearest base station"), we can answer questions that would otherwise be difficult or impossible like, "What is the distribution of the SINR for a typical network user?" We can extend this model to networks that operate at multiple frequencies or to networks with different types of base station as well.

\section{Discussion}
Stationary point processes are useful because they are relatively simple to work with and can be used to describe more complex statistical models. Those favorable properties come at the cost of making several simplifying assumptions. Firstly, stationary point processes cannot be defined on just a subset of $\mathbb{R}^2$. They must be defined over the whole set, so no real point process is truly stationary \footnote{That does not preclude being so close to stationary that the difference does not matter in practice.}. The variations in distribution with respect to position could cause a real process to deviate from the model. Another problem arises when the points are not actually independent of each other. For example, consider the first cellular network example where the points are base stations. If the base stations can interact, they are no longer independent and thus our model deviates from the real world. Other parts of the model can cause deviations from ideal as well. For example, most differential equations cannot be solved without recourse to approximations.

These problems do not mean that the PPP is useless in practice, but care must be taken when interpreting results. Whenever possible, the simulated results should be compared with data gathered from a real system. We saw this several times in the presentation. We can use this to improve the model incrementally. It may not be possible to make the deviation arbitrarily small but it might help us make it small enough for a particular application. For example, say we run some experiments with a basic PPP model and compare the results to some real data and find significant differences. If our analysis leads us to believe that some effect was not properly accounted for, this could be added and the simulations rerun. Other cases might require us to choose more accurate (and often more expensive) approximation methods if they exist.

All of the solutions offered are similar in that they seek to reduce the differences between the model and the problem domain. Sometimes changing the problem domain is useful. In the first example from the presentation, the model does not account for the use of frequency bands. 
\nocite{kham}
\nocite{randv}
\nocite{ppp}
\nocite{poissond}
\bibliographystyle{plain}
\bibliography{guestLectureAssignment}

\end{document}





